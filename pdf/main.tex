\documentclass[landscape,twocolumn,9pt]{jsarticle}
\usepackage[top=11mm,bottom=13mm,left=10mm,right=10mm,headsep=1mm,footskip=7mm]{geometry}
%\usepackage[top=10mm,bottom=10mm,left=10mm,right=10mm,headsep=0mm]{geometry}
\usepackage{amsmath,amssymb}
\usepackage{listings,jlisting,color}
\usepackage{fancyhdr}
\usepackage{txfonts}

% listings%{{{
\lstdefinelanguage[Competitive]{C++}[GNU]{C++}{
  morekeywords=[2]{
    repi,peri,rep,per,foreach
  }
}
\lstset{
  language=[Competitive]C++,
  basicstyle=\footnotesize\ttfamily,
  directivestyle= \color[rgb]{0.40,0.40,0.40}\textbf, % #666666
  keywordstyle=   \color[rgb]{0.00,0.40,0.60}\textbf, % #006699
  keywordstyle=[2]\color[rgb]{0.40,0.00,0.60}\textbf, % #660099
  commentstyle=   \color[rgb]{0.00,0.51,0.00},        % #008200
  stringstyle=    \color[rgb]{0.00,0.00,1.00},        % #0000ff
  showstringspaces=false,
  frame=single,
  framexleftmargin=18pt,
  xleftmargin=18pt,
  numbers=left,
  numbersep=1em,
  basewidth={0.54em,0.45em},
  lineskip=-2pt,
  tabsize=4,
}
%}}}

% fancyhdr%{{{
\pagestyle{fancy}
\renewcommand{\headrulewidth}{0.4pt}
\renewcommand{\footrulewidth}{0.4pt}
\renewcommand{\footruleskip}{0mm}
\fancyhfoffset{9pt}

\lhead{lyoz's library} \chead{} \rhead{}
\lfoot{} \cfoot{} \rfoot{last modified: \the\year/\the\month/\the\day}
%}}}

\begin{document}

\setlength{\abovedisplayskip}{1mm}
\setlength{\belowdisplayskip}{1mm}

\pagenumbering{roman}
\rhead{\thepage}

\tableofcontents
\clearpage

\pagenumbering{arabic}
\rhead{\thepage}

\section{環境}%{{{
\subsection{.Xmodmap}
[変換]と[半角/全角],[無変換]と[Escape]を入れ替え,[CapsLock]を[Ctrl]に割り当てる.
\lstinputlisting[language=]{../misc/Xmodmap.txt}

\subsection{.vimrc}
不要そうなものは適当に削る.
\lstinputlisting[language=]{../misc/vimrc.txt}

\subsection{Makefile}
\lstinputlisting[language=]{../misc/Makefile.txt}
%}}}

\section{テンプレート}%{{{
\subsection{C++}
\lstinputlisting{../misc/template.cpp}

\subsection{Java}
\texttt{Solver\#main}に書く.
\lstinputlisting[language=Java]{../misc/template.java}
%}}}

\section{グラフ}%{{{

\subsection{定義}
\begin{table}[h]
\begin{tabular}{ll}
マッチング &端点を共有しない辺集合 \\
独立集合   &どの2点も隣接しない頂点集合 \\
クリーク   &どの2点も隣接している頂点集合 \\
支配集合   &自身とその隣接頂点のみで全頂点を被覆する頂点集合 \\
辺被覆     &どの頂点も少なくとも1つの辺に接続している辺集合 \\
点被覆     &どの辺も少なくとも1つの頂点に接続している頂点集合 \\
\end{tabular}
\end{table}

\subsection{定理}
グラフ$G=(V,E)$について,
\begin{align*}
|最大マッチング|+|最小辺被覆|=|V| \\
  |最大独立集合|+|最小点被覆|=|V|
\end{align*}

二部グラフであるとき,
\begin{align*}
|最大マッチング|=|最小点被覆| \\
  |最大独立集合|=|最小辺被覆|
\end{align*}

\paragraph{Dilworthの定理・Mirskyの定理\\}\indent
推移性をみたすDAGであるとき,
\begin{center}\tabcolsep=0mm
\begin{tabular}{rl}
Dilworthの定理: &濃度最大の反鎖の濃度=個数最小の鎖被覆の個数 \\
Mirskyの定理:   &濃度最大の鎖の濃度=個数最小の反鎖被覆の個数
\end{tabular}
\end{center}

\paragraph{DAGの最小独立パス被覆}
\[ \text{(最小独立パス被覆)}=|V|-\text{($V+V'$を頂点とする二部グラフの最大マッチング)} \]

DAGが推移性をもつとき,
\[ \text{(最小独立パス被覆)}=\text{(最小パス被覆)} \]

\subsection{基本要素}
\lstinputlisting{../graph/graph.cpp}

\subsection{単一始点最短路(Dijkstra)}
計算量:$O(E\log V)$
\lstinputlisting{../graph/dijkstra.cpp}

\subsection{単一始点最短路(Bellman-Ford)}
計算量:$O(VE)$
\lstinputlisting{../graph/bellmanford.cpp}

\subsection{全点間最短路(Warshall-Floyd)}
計算量:$O(V^3)$
\lstinputlisting{../graph/warshallfloyd.cpp}

\subsection{最小全域木(Prim)}
\texttt{root}を含む最小全域木を計算する.\texttt{tree}に構成辺を保存し,重みの総和を返す.

計算量:$O(E\log V)$
\lstinputlisting{../graph/prim.cpp}

\subsection{最小全域森(Kruskal)}
\texttt{forest}に構成辺を保存し,重みの総和を返す.

計算量:$O(E\log V)$
\lstinputlisting{../graph/kruskal.cpp}

\subsection{最小シュタイナー木(Dreyfus-Wagner)}
与えられた隣接行列とターミナルの集合に対し,最小シュタイナー木の重みの総和を返す.

計算量:$O(3^TV+2^TV^2+V^3)$
\lstinputlisting{../graph/dreyfuswagner.cpp}

\subsection{最大独立集合}
計算量:$O(1.466^n n)$
\lstinputlisting{../graph/maxindependentset.cpp}

\subsection{トポロジカルソート}
計算量:$O(V+E)$
\lstinputlisting{../graph/topologicalsort.cpp}

\subsection{強連結成分分解}
計算量:$O(V+E)$
\lstinputlisting{../graph/scc.cpp}

\subsection{橋}
計算量:$O(V+E)$
\lstinputlisting{../graph/bridge.cpp}

\subsection{関節点}
計算量:$O(V+E)$
\lstinputlisting{../graph/articulation.cpp}

\subsection{2-SAT}
$u\lor v \Leftrightarrow(\overline u\to\overline v)\land(\overline v\to\overline u)$
と式変形し,二項関係$\to$で辺を張り強連結成分分解する.強連結成分の真偽は一致するため,
$x$と$\overline x$が同じ強連結成分に属するならば充足不能.そうでないならば充足可能.

\subsection{最小共通先祖}
計算量:構築$O(V \log V)$,クエリ$O(\log V)$
\lstinputlisting{../graph/lca.cpp}

\subsection{二部性判定}
計算量:$O(V+E)$
\lstinputlisting{../graph/isbipartite.cpp}
%}}}

\section{フロー}%{{{
\subsection{最大流(Dinic)}
計算量:$O(V^2E)$
\lstinputlisting{../flow/dinic.cpp}

\subsection{最小費用流}
計算量:$O(FE\log V)$
\lstinputlisting{../flow/mincostflow.cpp}

\subsection{2部グラフの最大マッチング(na\"\i ve)}
計算量:$O(VE)$
\lstinputlisting{../flow/bipartitematching.cpp}

\subsection{2部グラフの最大マッチング(Hopcroft-Karp)}
計算量:$O(\sqrt{V}E)$
\lstinputlisting{../flow/hopcroftkarp.cpp}

\subsection{DAGの最小独立パス被覆}
\lstinputlisting{../flow/minpathcover.cpp}

\subsection{無向グラフの全域最小カット(Nagamochi-Ibaraki)}
グラフが連結でないとき0を返す.
計算量:$O(V^3)$
\lstinputlisting{../flow/nagamochiibaraki.cpp}
%}}}

\section{文字列}%{{{
\subsection{Knuth-Morris-Pratt}
最短周期は\texttt{n-fail[n]}で求まる.

計算量:テキスト長$N$,パターン長$M$として,初期化$O(M)$,検索$O(M+N)$
\lstinputlisting{../string/kmp.cpp}

\subsection{最長回文(Manacher)}
計算量:$O(N)$
\lstinputlisting{../string/longestpalindrome.cpp}

\subsection{Suffix Array(K\"arkk\"ainen-Sanders)}
計算量:構築$O(N)$,LCPの計算$O(N)$.
\lstinputlisting{../string/kaerkkaeinensanders.cpp}

\subsection{Suffix Array(SA-IS)}
計算量:構築$O(N)$,LCPの計算$O(N)$.ただし$|\Sigma|=O(1)$とする.
\lstinputlisting{../string/sais.cpp}
%}}}

\section{動的計画法}%{{{
\subsection{最長共通部分列}
計算量:$O(MN)$
\lstinputlisting{../dp/lcs.cpp}

\subsection{最長増加部分列}
計算量:$O(N\log N)$.
\lstinputlisting{../dp/lis.cpp}
%}}}

\section{データ構造}%{{{
\subsection{Union-Find Tree}
計算量:各操作に関して$O(\alpha(N))$.ただし$\alpha(N)$はアッカーマン関数の逆関数.
\lstinputlisting{../datastructure/unionfind.cpp}

\subsection{Fenwick Tree}
計算量:各操作に関して$O(\log N)$
\lstinputlisting{../datastructure/fenwicktree.cpp}

\subsection{Segment Tree}
計算量:初期化$O(N)$,更新$O(\log N)$,クエリ$O(\log N)$
\lstinputlisting{../datastructure/segmenttree.cpp}

\subsection{Range Tree}
計算量:初期化$(O(N \log^2 N))$,クエリ$O(N \log^2 N)$
\lstinputlisting{../datastructure/rangetree.cpp}
%}}}

\section{数学}%{{{
\subsection{数学定数}
\paragraph{円周率} \hspace{12.16mm}
$ \pi=   3.1415926535897932384626433832795029 $
\paragraph{自然対数の底} \hspace{2.66mm}
$ e=     2.7182818284590452353602874713526625 $
\paragraph{オイラーの定数}
$ \gamma=0.5772156649015328606065120900824024 $

\subsection{ユークリッドの互除法}
\texttt{ExtendedGCD}は$ax+by=\gcd(a,b)$をみたす$x,y$を求める.
計算量:$O(\log\max(a,b))$
\lstinputlisting{../math/euclidian.cpp}

\subsection{逆元}
$ax\equiv 1\; (\mathrm{mod}\; m)$をみたす$x$を求める.
逆元が存在しない場合は0を返す.
\lstinputlisting{../math/modinverse.cpp}

\subsection{篩}
エラトステネスの篩と,それを用いた区間篩.
\paragraph{計算量} エラトステネスの篩:$O(N\log\log N)$,区間篩:$O(\sqrt b \log\log b+(b-a)\log\log b)$
\lstinputlisting{../math/sieve.cpp}

\subsection{素数判定(Miller-Rabin)}
問題によって実装を切り替える.
\lstinputlisting{../math/millerrabin.cpp}

\subsection{Gauss-Jordanの消去法}
解が一意でないとき\texttt{false}を返す.
計算量:$O(N^3)$
\lstinputlisting{../math/gaussjordan.cpp}
%}}}

\section{幾何}%{{{
\subsection{基本要素}
\lstinputlisting{../geometry/geometry.cpp}

\subsection{回転方向}
\lstinputlisting{../geometry/ccw.cpp}

\subsection{面積}
\lstinputlisting{../geometry/area.cpp}

\subsection{交差判定}
\lstinputlisting{../geometry/intersect.cpp}

\subsection{距離}
\lstinputlisting{../geometry/distance.cpp}

\subsection{円の接線}
\lstinputlisting{../geometry/tangent.cpp}

\subsection{凸包}
計算量:$O(N\log N)$
\lstinputlisting{../geometry/convexhull.cpp}

\subsection{凸多角形の包含判定}
\lstinputlisting{../geometry/cover.cpp}

\subsection{凸多角形の切断}
\lstinputlisting{../geometry/convexcut.cpp}

\subsection{線分アレンジメント}
同一線分上の推移的な辺を省略している.
計算量:$O(N^3)$
\lstinputlisting{../geometry/segmentarrangement.cpp}
%}}}

\section{その他}%{{{
\subsection{ビット演算}
\texttt{clz},\texttt{clzll},\texttt{ctz},\texttt{ctzll}は,$x=0$のときビット長を返す.
\lstinputlisting{../other/bitoperation.cpp}

\subsection{ハッシュ関数}
ユーザ定義型を\texttt{unordered\_set}や\texttt{unordered\_map}のキーとして使うためには,
\texttt{hash}の特殊化を書く必要がある.以下はFNV-1aによる実装例.
\lstinputlisting{../other/hash.cpp}

\subsection{Fairfieldの公式}
西暦1年1月0日からの経過日数を求める.$\text{(経過日数)}\equiv 0\; (\mathrm{mod}\; 7)$のとき日曜日.
\lstinputlisting{../other/fairfield.cpp}

\subsection{さいころ}
\lstinputlisting{../other/dice.cpp}
%}}}

\end{document}
